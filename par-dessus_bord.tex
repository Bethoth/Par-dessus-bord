% Set up the document's format to A4 and the font's size to 12pt.
\documentclass[a4paper]{report}
\usepackage[fontsize=14.5pt]{scrextend}

% Set up the input's encoding to UTF-8, the document's font and language to T1 (adapted to french) and french (the grammar linter uses this parameter).
\usepackage[utf8]{inputenc}
\usepackage[T1]{fontenc}
\usepackage[french]{babel}

% Set up the document's margins.
\usepackage{geometry}
\geometry{hmargin=1.5cm,vmargin=1.5cm}

% Set up the document's title, author and date.
\title{Par-dessus bord\\\large Forme hyper-brève}
\author{Michel Vinaver (adapté par Romain Bricout)}
\date{\today}

% The three main maths packages. They are used for a lot of things.
\usepackage{amssymb,amsmath}
\usepackage{mathtools}

% Useful to create nice and easy signs or variations tables.
\usepackage{tkz-tab}

% Useful to create any kind of visual representation (graph functions, illustrate geometry problems, etc)
\usepackage{tikz}

% Allows to edit the itemize environment's default item document-wide.
\usepackage{enumitem}

% Allows to define \notfoo or \nfoo (not recommended) in order for \not\foo to work as wished.
\usepackage{newtxmath}

% Makes the table of contents clickable and gives useful commands for links in general.
\usepackage{hyperref}
\hypersetup{colorlinks=false,linktoc=all}

% Gives the llbracket and rrbracket commands for integer intervals.
\usepackage{stmaryrd}

% Useful to insert nice-looking quotes.
\usepackage{epigraph}

% Allows to insert chapter-specific table of contents.
\usepackage{minitoc}
\mtcselectlanguage{french}
\setcounter{minitocdepth}{1}

% Allows to insert boxes.
\usepackage[framemethod=TikZ]{mdframed}
\mdfsetup{skipabove=\topskip,skipbelow=\topskip}

% Useful when units are needed.
\usepackage{siunitx}

% Allows to customize chapters, sections, etc
\usepackage{titlesec}

% Set up the horizontal space before the first line of a new paragraph to 2em and the vertical space between two paragraphs to 1em.
\setlength{\parindent}{2em}
\setlength{\parskip}{1em}

% Adds 0.5em to the vertical space between two lines in an align environment. It looks better.
\addtolength{\jot}{0.5em}

% Allows align environment to break if it's too long to fit in the page where it began.
\allowdisplaybreaks[1]

% Trick to make semicolons considered like relation operators (such as =) and therefore being equidistantly spaced from the two numbers around it.
\mathcode`;=\numexpr\mathcode`;-"3000

% Commands for size-adaptative parentheses, brackets, curly brackets, absolute value and magnitude.
\newcommand{\paren}[1]{\left(#1\right)} % (x)
\newcommand{\croch}[1]{\left[#1\right]} % [x]
\newcommand{\accol}[1]{\left\lbrace#1\right\rbrace} % {x}
\newcommand{\abs}[1]{\left\lvert#1\right\rvert} % |x|
\newcommand{\norme}[1]{\left\|#1\right\|} % ||x||

% Commands for size-adaptative intervals and integer intervals. The commands' roots are "interv" and "interventier" and the added e or i at the end mean "excluded" and "included" respectively.
\newcommand{\intervii}[2]{\left[#1;#2\right]} % [a;b]
\newcommand{\intervee}[2]{\left]#1;#2\right[} % ]a;b[
\newcommand{\intervie}[2]{\left[#1;#2\right[} % [a;b[
\newcommand{\intervei}[2]{\left]#1;#2\right]} % ]a;b]
\newcommand{\interventierii}[2]{\left\llbracket#1;#2\right\rrbracket} % non-ASCII characters needed
\newcommand{\interventieree}[2]{\left\rrbracket#1;#2\right\llbracket} % non-ASCII characters needed
\newcommand{\interventierie}[2]{\left\llbracket#1;#2\right\llbracket} % non-ASCII characters needed
\newcommand{\interventierei}[2]{\left\rrbracket#1;#2\right\rrbracket} % non-ASCII characters needed

% Commands for usually used sets.
\newcommand{\N}{\mathbb{N}} % natural integers
\newcommand{\Z}{\mathbb{Z}} % relative integers
\newcommand{\D}{\mathbb{D}} % decimal numbers
\newcommand{\Q}{\mathbb{Q}} % rational numbers
\newcommand{\R}{\mathbb{R}} % real numbers
\newcommand{\C}{\mathbb{C}} % complex numbers
\newcommand{\U}{\mathbb{U}} % complex numbers whose modulus is 1
\renewcommand{\P}{\mathbb{P}} % prime numbers (\P normally prints a pilcrow : ¶)
\renewcommand{\H}{\mathbb{H}} % quaternions (\H normally prints slanted quotation marks)
\renewcommand{\O}{\mathbb{O}} % octonions (\O normally prints a slashed capital o : Ø)
\newcommand{\M}{\mathcal{M}} % matrices
\newcommand{\GL}{\mathrm{GL}} % invertible matrices
\renewcommand{\S}{\mathcal{S}} % solutions of an equation (\S normally prints a silcrow : §)

% Redefines \Re and \Im to print Re and Im (the same way as ln or lim) instead of fraktur R and I which don't look nice and are less readable.
\renewcommand{\Re}{\operatorname{Re}}
\renewcommand{\Im}{\operatorname{Im}}

% Command to print an upright e for the exponential instead of a slanted e and put the exponent.
\newcommand{\e}[1]{\mathrm{e}^{#1}}

% Command to print the imaginary i with a little space on the right. This way, the exponents don't look confusing. \i normally prints a dotless i.
\renewcommand{\i}{i\mkern1mu}

% Redefines \vec such that the arrow covers the whole name of the vector.
\renewcommand{\vec}[1]{\overrightarrow{#1}}

% Commands for 2D and 3D vectors' coordinates
\newcommand{\dcoords}[2]{\begin{pmatrix}#1\\#2\end{pmatrix}}
\newcommand{\tcoords}[3]{\begin{pmatrix}#1\\#2\\#3\end{pmatrix}}

% Redefines binom to print nicer parentheses around the numbers.
\renewcommand{\binom}[2]{\begin{pmatrix}#2\\#1\end{pmatrix}}

% Command for a QED black square. It automatically prints a whitespace before the square such that it looks nice.
\newcommand{\cqfd}{\text{ }\blacksquare}

% Commands with more explicit names for the best way to express divisibility (mid and nmid).
\newcommand{\divise}{\mid}
\newcommand{\notdivise}{\nmid}

% Commands that do the exact same thing but with explicit names for a complex number's conjugate and an event's negation in probability.
\newcommand{\conj}[1]{\overline{#1}}
\newcommand{\non}[1]{\overline{#1}}

% Command for a size-adaptative middle bar meaning "such that" (with spacing around it in order to look nice).
\newcommand{\tq}{\;\middle|\;}

% Command with an explicit name for the scalar product.
\newcommand{\scalaire}{\cdot}

% Shortcut for forcing displaystyle in inline mode.
\newcommand{\ds}{\displaystyle}

% Make the not version of implies, impliedby and iff look nice.
\newcommand{\notimplies}{\centernot{\implies}}
\newcommand{\notimpliedby}{\centernot{\impliedby}}
\newcommand{\notiff}{\centernot{\iff}}

% Shortcut for P(event).
\newcommand{\proba}[1]{P\paren{#1}}

% More explicit names for land (logical and) and lor (logical or).
\newcommand{\et}{\land}
\newcommand{\ou}{\lor}

% Explicitly named environment for tkz-tab tables. Automatically centers the table and handles the tikzpicture environment.
\newenvironment{tableau}
{
\begin{center}
\begin{tikzpicture}
}
{
\end{tikzpicture}
\end{center}
}

% More explicitly named commands for the creation of tkz-tab tables.
\newcommand{\tableauinit}[2]{\tkzTabInit{#1}{#2}}
\newcommand{\tableausignes}[1]{\tkzTabLine{#1}}
\newcommand{\tableauvariations}[1]{\tkzTabVar{#1}}

% Shortcut for the curve and the domain of the given function.
\newcommand{\courbe}[1]{\mathcal{C}_{#1}}
\newcommand{\ensembledef}[1]{\mathcal{D}_{#1}}

% Various environments that create boxes. Each one is one type of thing (example, proof, etc). Each type has its own automatic counter.
\newcounter{rem}[chapter]
\newenvironment{remarque}[1][]{%
\stepcounter{rem}%
\ifstrempty{#1}%
{\mdfsetup{%
frametitle={%
\tikz[baseline=(current bounding box.east),outer sep=0pt]
\node[anchor=east,rectangle,fill=black!20,rounded corners]
{\strut Remarque~\therem};}}
}%
{\mdfsetup{%
frametitle={%
\tikz[baseline=(current bounding box.east),outer sep=0pt]
\node[anchor=east,rectangle,fill=black!20,rounded corners]
{\strut Remarque~\therem~:~#1};}}%
}%
\mdfsetup{innertopmargin=10pt,linecolor=black!70,%
linewidth=2pt,topline=true,
frametitleaboveskip=\dimexpr-\ht\strutbox\relax,roundcorner=5pt}
\begin{mdframed}[]\relax%
}{\end{mdframed}}

\newcounter{def}[chapter]
\newenvironment{definition}[1][]{%
\stepcounter{def}%
\ifstrempty{#1}%
{\mdfsetup{%
frametitle={%
\tikz[baseline=(current bounding box.east),outer sep=0pt]
\node[anchor=east,rectangle,fill=black!20,rounded corners]
{\strut Définition~\thedef};}}
}%
{\mdfsetup{%
frametitle={%
\tikz[baseline=(current bounding box.east),outer sep=0pt]
\node[anchor=east,rectangle,fill=black!20,rounded corners]
{\strut Définition~\thedef~:~#1};}}%
}%
\mdfsetup{innertopmargin=10pt,linecolor=black!70,%
linewidth=2pt,topline=true,
frametitleaboveskip=\dimexpr-\ht\strutbox\relax,roundcorner=5pt}
\begin{mdframed}[]\relax%
}{\end{mdframed}}

\newcounter{exmp}[chapter]
\newenvironment{exemple}[1][]{%
\stepcounter{exmp}%
\ifstrempty{#1}%
{\mdfsetup{%
frametitle={%
\tikz[baseline=(current bounding box.east),outer sep=0pt]
\node[anchor=east,rectangle,fill=black!20,rounded corners]
{\strut Exemple~\theexmp};}}
}%
{\mdfsetup{%
frametitle={%
\tikz[baseline=(current bounding box.east),outer sep=0pt]
\node[anchor=east,rectangle,fill=black!20,rounded corners]
{\strut Exemple~\theexmp~:~#1};}}%
}%
\mdfsetup{innertopmargin=10pt,linecolor=black!70,%
linewidth=2pt,topline=true,
frametitleaboveskip=\dimexpr-\ht\strutbox\relax,roundcorner=5pt}
\begin{mdframed}[]\relax%
}{\end{mdframed}}

\newcounter{exo}[chapter]
\newenvironment{exercice}[1][]{%
\stepcounter{exo}%
\ifstrempty{#1}%
{\mdfsetup{%
frametitle={%
\tikz[baseline=(current bounding box.east),outer sep=0pt]
\node[anchor=east,rectangle,fill=black!20,rounded corners]
{\strut Exercice~\theexo};}}
}%
{\mdfsetup{%
frametitle={%
\tikz[baseline=(current bounding box.east),outer sep=0pt]
\node[anchor=east,rectangle,fill=black!20,rounded corners]
{\strut Exercice~\theexo~:~#1};}}%
}%
\mdfsetup{innertopmargin=10pt,linecolor=black!70,%
linewidth=2pt,topline=true,
frametitleaboveskip=\dimexpr-\ht\strutbox\relax,roundcorner=5pt}
\begin{mdframed}[]\relax%
}{\end{mdframed}}

\newcounter{sol}[chapter]
\newenvironment{solution}[1][]{%
\stepcounter{sol}%
\ifstrempty{#1}%
{\mdfsetup{%
frametitle={%
\tikz[baseline=(current bounding box.east),outer sep=0pt]
\node[anchor=east,rectangle,fill=black!20,rounded corners]
{\strut Solution~\thesol};}}
}%
{\mdfsetup{%
frametitle={%
\tikz[baseline=(current bounding box.east),outer sep=0pt]
\node[anchor=east,rectangle,fill=black!20,rounded corners]
{\strut Solution~\thesol~:~#1};}}%
}%
\mdfsetup{innertopmargin=10pt,linecolor=black!70,%
linewidth=2pt,topline=true,
frametitleaboveskip=\dimexpr-\ht\strutbox\relax,roundcorner=5pt}
\begin{mdframed}[]\relax%
}{\end{mdframed}}

\newcounter{theo}[chapter]
\newenvironment{theoreme}[1][]{%
\ifstrempty{#1}%
{\mdfsetup{%
frametitle={%
\tikz[baseline=(current bounding box.east),outer sep=0pt]
\node[anchor=east,rectangle,fill=black!20,rounded corners]
{\strut Théorème~\thetheo};}}
}%
{\mdfsetup{%
frametitle={%
\tikz[baseline=(current bounding box.east),outer sep=0pt]
\node[anchor=east,rectangle,fill=black!20,rounded corners]
{\strut Théorème~\thetheo~:~#1};}}%
}%
\mdfsetup{innertopmargin=10pt,linecolor=black!70,%
linewidth=2pt,topline=true,
frametitleaboveskip=\dimexpr-\ht\strutbox\relax,roundcorner=5pt}
\begin{mdframed}[settings={\refstepcounter{theo}}]\relax%
}{\end{mdframed}}

\newcounter{demo}[chapter]
\newenvironment{demonstration}[1][]{%
\stepcounter{demo}%
\ifstrempty{#1}%
{\mdfsetup{%
frametitle={%
\tikz[baseline=(current bounding box.east),outer sep=0pt]
\node[anchor=east,rectangle,fill=black!20,rounded corners]
{\strut Démonstration~\thedemo};}}
}%
{\mdfsetup{%
frametitle={%
\tikz[baseline=(current bounding box.east),outer sep=0pt]
\node[anchor=east,rectangle,fill=black!20,rounded corners]
{\strut Démonstration~\thedemo~:~#1};}}%
}%
\mdfsetup{innertopmargin=10pt,linecolor=black!70,%
linewidth=2pt,topline=true,
frametitleaboveskip=\dimexpr-\ht\strutbox\relax,roundcorner=5pt}
\begin{mdframed}[]\relax%
}{\end{mdframed}}

\begin{document}
%\renewcommand{\labelitemi}{\(\bullet\)}

\maketitle

\renewcommand{\contentsname}{Sommaire}
\dominitoc\tableofcontents

\chapter*{Préface}
\addcontentsline{toc}{chapter}{Préface}

\begin{center}
DIALOGUE AVEC MOI-MÊME\\(APOCRYPHE)
\end{center}

-- Vous dessinez quoi dans ce bloc note ? On dirait un grand bâtiment. En réunion, je vous vois toujours en train de dessiner quelque chose.

-- Je fais du \textit{doodling}. Ce qui vient à la pointe de mon crayon. N'importe quoi. Ça m'aide à penser.

-- Qu'est-ce que ce \textit{building} représente ?

-- C'est la tour à Boston dans laquelle il y a le \(57^{\text{e}}\) étage, celui des bureaux du siège de l'entreprise. Tous les grands patrons de Gillette au plan mondial y ont leur bureau avec les secrétaires. La moquette y est épaisse. Ce sont des lieux sans bruit.

-- J'imagine que vous n'y avez jamais mis les pieds.

-- Faux. Une fois ils m'ont convoqué pour se faire une idée de moi. C'est une de leurs fonctions que d'identifier, parmi les jeunes cadres, les futurs responsables de leurs filiales dans les différents pays et de programmer leur formation. Ainsi, ils me mirent sur la route. J'accompagnais des représentants Gillette dans la tournée de leur clientèle en Grande-Bretagne, en France. A la suite de cette initiation, je fus nommé à la tête de Gillette Benelux, puis de Gillette Italie. Enfin, le poste suprême en France devint vacant et me fut attribué, qui comportait des directions générales et commerciales, et la direction des trois usines en Haute-Savoie. Avec femme et enfants à présent, je continuais d'écrire. Allais-je pouvoir continuer longtemps à cheval sur deux métiers plus une famille ?

Je négociai ma sortie de l'opérationnel chez Gillette, en n'y conservant que des fonctions légères et un salaire allégé lui aussi. Ce faisant, je commençai à me rendre compte que mon activité d'écriture théâtrale fuitait dans l'entreprise parmi le personnel. Mon image s'y brouillait.

Le temps était venu d'une œuvre majeure, une œuvre libérée de toutes les précautions. Une œuvre dans l'écriture de laquelle j'avancerais sans masque, sans défense, où je ne serais plus divisé. J'allais jeter par-dessus bord toutes les convenances et règles qui font les bonnes pièces. La pièce aurait pour nom, eh bien oui, \textit{Par-dessus bord}. L'entreprise fabriquerait et vendrait un produit un peu trivial, le PQ comme on le désigne familièrement. Son patron et propriétaire aurait deux fils, l'un bien né, l'autre bâtard. Le bâtard serait résolu à se débarrasser de son père et de son frère, par tous les moyens, pour rester seul à bord. L'épouse du bien né, américaine, souhaiterait ardemment que la collection lui revienne en temps voulu. Voilà. Quelques détails comme ça. Ah, et la pièce compterait parmi ses personnages un certain Passemar (un nom qui avait avivé ma sympathie et ma curiosité), retraité, habitant seul dans une lointaine banlieue, bricoleur à ses heures dans la cave de son pavillon, et qui avait breveté une technique pour déposer une fine couche de téflon sur les tranchants d'une lame de rasoir de sûreté afin d'en exalter la douceur.

La pièce serait aussi un laboratoire de recherche ! Mais son domaine de recherche serait l'innovation sur le plan de l'écriture ! Fin de la continuité. Fin de la ponctuation. Fin de la distinction entre les genres. Fin de tous les interdits. Bouclages. Contrepoint. Entrelacs. Fulgurances. Répétitions/variations.

Ce qui avait le plus intéressé Planchon dans la pièce, quand je lui en ai donné le manuscrit à lire, c'est qu'il y trouvait la première vision claire, détaillée, au théâtre, du fonctionnement du système capitaliste. Planchon n'était pas insensible, simultanément, à la vogue de la comédie musicale dans le cinéma américain. Il a voulu marier les deux. Sa création de \textit{Par-dessus bord}, au TNP à Villeurbanne puis à l'Odéon à Paris, fut chaleureusement reçue. La mise en scène en a été un spectacle mémorable, superbe, frénétique. Mais qui n'a troublé ni inquiété personne. Son charme l'a emporté sur sa pugnacité critique. Mon objectif qui était de me faire virer a échoué.

Que dire aujourd'hui de \textit{Par-dessus bord} ? J'en ai fait trois réductions, la \textit{Brève}, la \textit{Super-brève} et l'\textit{Hyper-brève}, et j'en arrive à ne pas avoir de préférence entre mes quatre versions. J'ai tenu à ce qu'elle soient toutes publiées.

Il m'est même arrivé d'autoriser des versions encore plus réduites et en même temps plus engagées dans la contestation, par exemple, au Vigan dans les Cévennes ardéchoises, en 2005, sous la direction d'un prof d'histoire et géographie.

\textit{Par-dessus bord} est en quelque sorte ma pièce matricielle. Toutes mes pièces qui ont suivi peuvent se réclamer d'elle, des chemins qu'elle a tracés. Avec elle, je me libérais aussi d'un tabou, de l'interdiction que je m'étais imposée de prendre l'entreprise, ou plus généralement le travail, comme un sujet de mes pièces. Je mettrais fin à la séparation entre les deux plans sur lesquels se déroule ma vie, celui du jeune cadre débutant dans une entreprise commerciale et industrielle et celui du jeune romancier-dramaturge.

\begin{flushright}
MICHEL VINAVER
\end{flushright}
\end{document}

















